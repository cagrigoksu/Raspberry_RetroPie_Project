\chapter{Configurations}
\label{cha:configurations}

It is necessary to configure inputs and interface settings to ensure proper operation. This chapter explains how to map keyboard and joystick inputs to default setup, adjust UI modes, and manage game visibility.

\section{Emulator Input Mapping}
\label{sec:input_mapping}

Input mapping allows the emulator to recognize user actions correctly, whether from a keyboard or physical arcade controls. The menu to configure buttons is accessible in EmulationStation screen, by pressing the assigned \textbf{\texttt{Start}} button to enter the main menu and selecting \textbf{\texttt{Configure Input}}. In order to keep the consistency between multiple arcade machines, please \underline{\textbf{use the same button mapping}} as indicated below.

\subsection{Keyboard}
\label{subsec:keyboard}

Keyboard inputs are mainly used for configuration and system navigation. To set up keyboard controls:

\begin{enumerate}
  \item From the RetroPie main menu (Enter button as default), navigate to \textbf{Configure Input}.
  \item Follow the on-screen prompts to assign keys for system functions, hotkeys, and navigation, \textbf{\underline{as follows}}.
    \begin{itemize}
      \item D-Pad Up, Down, Left, Right → Up, Down, Left, Right
      \item Start → Enter
      \item Select → Right Shift
      \item Button-A → X
      \item Button-B → Z
      \item Button-X → S
      \item Button-Y → A
      \item Left Shoulder → Q
      \item Right Shoulder → W
      \item Hotkey → Left Ctrl
    \end{itemize}
\end{enumerate}

These are some useful key or cobinations:
\begin{itemize}
  \item Hotkey + Enter: Exit to EmulationStation
  \item F4: Enter terminal when on EmulationStation
\end{itemize}

\subsection{Joystick and Buttons}
\label{subsec:joystick_buttons}

Joystick and arcade button mapping is used exclusively for gameplay. RetroPie automatically detects the controllers on startup. To configure:

\begin{enumerate}
  \item From the RetroPie main menu (Enter button as default), navigate to \textbf{Configure Input}.
  \item Follow the on-screen prompts to assign keys for system functions, hotkeys, and navigation, \textbf{\underline{as follows}}.
    \begin{itemize}
      \item D-Pad Up, Down, Left, Right → Up, Down, Left, Right on joystick
      \item Start → Start
      \item Select → Select
      \item Button-A → A
      \item Button-B → B
      \item Button-X → X
      \item Button-Y → Y
      \item Hotkey → \textbf{Important:} Do not enter any button for hotkey by skipping it. When you exit the configuration, a hotkey button will be asked to use Select button as default. Choose \textbf{Yes} if you have multiple games, \textbf{No} if you have single game.
    \end{itemize}
\end{enumerate}

\section{Emulator UI Settings}
\label{sec:emulator_settings}

These settings help secure configurations and make the arcade machine user-friendly. Using the settings below, the configuration menus will be only accessible by keyboard. The gamers are prevented from entering these menus.

\subsection{UI Mode Settings}
\label{subsec:ui_settings}

This setting manages access to the menus by the user. To configure:

\begin{enumerate}
  \item From the main menu (Enter button), go to \textbf{UI Settings}. Under UI Mode, there are three different modes:
    \begin{itemize}
      \item \textbf{Full Mode} – Complete access to all menus and settings.
      \item \textbf{Kiosk Mode} – Simplified interface hiding advanced menus.
      \item \textbf{Kid Mode} – Restricts access to configuration screens to prevent accidental changes by gamers.
    \end{itemize}

  \item Choose the mode: Full mode to configure system, Kid mode before end-user access to machine.
  \item To return to Full mode while in Kid mode, press Up, Up, Down, Down, Left, Right, Left, Right, B, A buttons in order.
\end{enumerate}

\subsection{Game Settings}
\label{subsec:game_settings}

Game metadata determines which games are visible in each UI mode:

\begin{enumerate}
  \item Navigate to game in game list and open \textbf{Options} by pressing Select button.
  \item Edit game metadata and make \textbf{KIDGAME} on.
    \begin{itemize}
      \item Only games tagged appropriately for Kid mode will appear when the arcade is   running in that restricted mode, ensuring children access approved games.
      \item Full mode will display the complete game library.
    \end{itemize}
\end{enumerate}
