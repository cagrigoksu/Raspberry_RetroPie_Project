\chapter{SD Card Preparation}
\label{cha:sd_card_preparation}

Before the arcade machine can run, the Raspberry Pi needs a bootable SD card with the RetroPie image. This chapter explains how to extract an image from an existing SD card for backup and how to write a new image to an SD card. Instructions are provided for both Windows and Linux-based systems (e.g., Ubuntu, Debian, ...).  

\section{Extracting Images from SD Card}
\label{sec:extracting_images_sd_card}

Extracting an image from an SD card creates a complete backup of the system. The extracted image can be written to an SD card in order to run the games on multiple arcade machines. This method is also useful to keep a copy in case of corruption.  

\subsection{Windows}
\label{subsec:extracting_windows}

On Windows, the simplest tool for extracting SD card images is \textbf{Win32 Disk Imager}.  
\begin{enumerate}
  \item Download and install Win32 Disk Imager.  
  \item Insert the SD card into your computer using an SD card reader.  
  \item Open the program and select the drive letter that corresponds to your SD card.  
  \item Choose a location and filename for the image file (\texttt{.img}).  
  \item Click \textbf{Read} to copy the SD card contents into the image file.  
\end{enumerate}

\subsection{Linux}
\label{subsec:extracting_linux}

On Linux-based systems, SD card images can be extracted using the \texttt{dd} command.  
\begin{enumerate}
  \item Insert the SD card into your computer.  
  \item Identify the device name using \texttt{lsblk} (e.g., \texttt{/dev/sda}). \textbf{Note:} device name can be different than \texttt{/dev/sdX} depending on your operating system  
  \item Run the following command in a terminal:  
  \begin{verbatim}
  sudo dd if=/dev/sdX of=~/backup.img bs=4M status=progress
  \end{verbatim}  
  %\item compress the image
\end{enumerate}

Replace \texttt{sdX} in \texttt{/dev/sdX} with the correct device name, and then wait for the process to complete. In case of an interruption during the process, retry step 3.  

\section{Writing Images to SD Card}
\label{sec:writing_images_sd_card}

Writing an image to the SD card installs the RetroPie system. This process erases all data on the card, so ensure that any important files are backed up before proceeding.  

\subsection{Windows}
\label{subsec:writing_windows}

On Windows, tools like \textbf{balenaEtcher} or \textbf{Win32 Disk Imager} can be used.  
\begin{enumerate}
  \item Download and install balenaEtcher (recommended for simplicity).  
  \item Insert the SD card into your computer.  
  \item Open balenaEtcher and select the RetroPie image file (\texttt{.img} or \texttt{.zip}).  
  \item Select the target SD card.  
  \item Click \textbf{Flash} and wait for the process to finish.  
\end{enumerate}

\subsection{Linux}
\label{subsec:writing_linux}

On Linux-based systems, an image can be written with the \texttt{dd} command.  
\begin{enumerate}
  \item Insert the SD card into your computer.  
  \item Identify the device name using \texttt{lsblk} (e.g., \texttt{/dev/sda}). \textbf{Note:} device name can be different than \texttt{/dev/sdX} depending on your operating system
  \item Use the \texttt{dd} command:  
  \begin{verbatim}
  sudo dd if=retropie.img of=/dev/sdX bs=4M status=progress conv=fsync
  \end{verbatim}   
\end{enumerate}

Replace \texttt{retropie.img} with the image filename and \texttt{/dev/sdX} with your SD card device. Once the process completes, safely eject the SD card. This SD card is ready to be used to boot up the Raspberry Pi with RetroPie. 
