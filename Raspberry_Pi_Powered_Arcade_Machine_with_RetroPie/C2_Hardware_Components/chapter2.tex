\chapter{Hardware Components}
\label{cha:hardware_components}

This chapter outlines the essential hardware components needed to build a fully functional arcade machine using a Raspberry Pi and RetroPie. From the display to the input controls, each piece plays a critical role in delivering the project.

\section{General Hardware Components}
\label{sec:general_hardware_components}

The core of the arcade machine on Raspberry Pi relies on a few key hardware elements that bring the system to life. 

\begin{itemize}
    \item Raspberry Pi 3B+ 
    \item Screen (Raspberry Pi 3B+ supports HDMI)
    \item 5V-3A micro-USB power supply (to power Raspberry Pi)
    \item SD card (to store RetroPie and game ROMs)
    \item Input devices (keyboard, joystick, and buttons)    
    \item Speaker (3.5mm audio jack, preferably USB-powered, optional)
\end{itemize}

\section{Input Controls}
\label{sec:input_controls}

The core of the arcade machine lies in its input controls, which give the ability to configure the system and play the games. The joysticks, mounted securely to the control panel, enable precise directional input to navigate menus and gameplay. The arcade buttons are arranged to handle actions such as jumping, shooting, or selecting options. A USB encoder connects these controls (joystick and buttons) to the Raspberry Pi, translating physical inputs into digital signals that RetroPie can interpret. The keyboard input is used solely to configure the machine. 

%\subsection{Keyboard}
%\label{subsec:keyboard}
%\subsection{Joysticks}
%\label{subsec:joysticks}

%\subsection{Buttons}
%\label{subsec:buttons}

%\section{Wiring and Connectors}
%\label{sec:wiring_connectors}
%\subsection{Raspberry Pi}
%\label{subsec:raspberry_pi}
%\subsection{USB-Encoders}
%\label{subsec:usb_encoders}
